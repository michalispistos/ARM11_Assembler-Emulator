\documentclass[a4paper]{article}
\usepackage[utf8]{inputenc}

\title{ARM Final Report}
\author{Sarveen Santhiramohan (Group Leader) \\ Anthony Lee\\ Ezra Sitorus\\ Michalis Pistos}
\date{May/June 2020}

\begin{document}
\maketitle

\bigskip
\subsection*{1. Structure and implementation of assembler}

Our assembler consists of a main function in assemble.c which calls a series of functions which
are defined in the library assemble{\_}utils. 

To implement the assembler, we decided on using the two pass method with a map as the symbol table.
Initially, our map was only used to store label addresses and opcodes for instructions. Each
key (word) was mapped to an integer value (code) corresponding to the address or opcode of the key.
However, at a later stage, we decided on adding another field for a function pointer so that instructions
of different types could be mapped to the respective assemble functions in the second pass. Each map node holds
these values as well as a pointer to the next node. The map structure is then composed of 2 different map node sets
as well as an integer holding the value of the last instruction address. The first set of map nodes is for general 
purpose, holding most of the instruction/label strings and code/location integers. The second is used in single data
transfer instructions, when large values need to be stored.
\par
In the first pass, we create the symbol table and add the addresses of each label that we encounter.
Before executing the second pass, we add every instruction that we support along with their
respective opcodes and functions to the symbol table. We also add a pointer to the 'end' of the program
which is used in single data transfer instructions where the expression is required to be stored at the 
end of the assembly file if it is too large. The first pass is done in the function create{\_}symbol{\_}table.
\par
We then proceed to the second pass, in which we directly translate each assembly instruction into a 32 bit binary code.
This is done by looking at the first word of each line which determines its instruction type by looking up through the 
lookup table. We pass the entire line then to the appropriate function and outputs a 32 bit instruction that gets written 
into the ouput file. Labels are ignored and when a single data transfer requires storing a large number we can use the end value 
to find the last address of the instructions and place the new number there. This is done by adding a new element
to the map with no string and the large number required as the integer code. If multiple numbers are added,
these entries are added in a queue-like manner to the end of the map which can be added once we finish 
translating the original assembly code.
\par
In both the first pass and the second pass, we used the tokenizer function to parse the assembly files. The tokenizer
created an array of character pointers (strings) from one line of assembly. For example "add r1 \#2" would be split
into ["add","r1","\#2"]. In the second pass, we used the get function of our map on the first element of the array
returned by tokenizer to retrieve a function pointer so that we could pass the tokens to that function. We decided
to do it this way with separate assemble functions for each data type for two reasons. Firstly, this way everyone
could work on a function each without interfering. Secondly, we thought that making it separate would make our
code more readable and easier to modify in the event we needed to make changes to it in the future. The results from
the assemble{\_}functions were then written to an array which then is written to a binary file by the function write{\_}file.


\bigskip
\bigskip

\subsection*{2. Testing of implementation}
We mainly just used the provided test-suite for our testing as it was already quite extensive, including many corner cases. 
However, before our code reached the stage where we could use the given test suite to test it, we ran some test files individually 
using the gcc to compile a prototype to check, for example, if in Part I our code could initially read the binary file. We also did more frequent 
tests of different parts of the program so that any fixing would be minimised. This was also quite helpful as we could
rely on the completed parts to perform (mostly) correct operations for other parts of the program that depended on it. 

We also ran our program with gdb including the -g flag to spot errors in our code and help us debug it. When certain test cases failed,
gdb was very useful to check if variables held the correct value. This also meant that we could pinpoint the specific line of code
which made the program do the wrong operation.

To check for memory leaks, we ran our code with Valgrind, using --leak-check=full and --show-leak-kinds=all flags 
with the test files to ensure that we have freed everything that we have allocated memory to in the heap. We had an extensive Valgrind
testing session towards the end to make sure that both our assembler and emulator were free of memory leaks. We made sure to test on 
both our own computers and also the lab machines in order to ensure that our program would not time out, even on slower computers. 

Finally, we created extra programs to test the functionality of our program. These extra tests also perform realistic applications. The
first program - 4digitpalindrome - is a program to verify a simple number theory idea: all 4 digit palindromes are divisible by 11. We 
verify this for the number 3443. This is done by dividing via repeated subtraction. We can observe that the reminder when divided by 11 is 0,
which is the value stored in r0. The second program - factorial - builds on the factorial test case. We calculate 10! via repeated multiplication
and also test out single data transfer by storing the result into 0x100 in memory. The next program is geometric - which calculates the geometric 
series, using 2 as the ratio and store the result into memory as well. We use a counter (r5) to repeated multiply a value by 2, and keep adding 
this to a sum (r4). The result is 63, which is $2^{6}-1$. This value is stored into memory as well. quotRem is similar to the quotRem function in 
Haskell, we calculate the quotient and remainder when 163 is divided by 34. This program operates similarly to the 4digitpalindrome program, through 
through repeated subtraction until a condition is satisfied (the 'remainder' is less than the divisor). r0 contains the remainder and r1 contains the 
quotient. We can verify that the program is correct as 163 = 4 x 34 + 27. These values are stored into memory as well (consecutive words as well). 
The final program is fibonacci, which we use to calculate the 21st fibonacci number via a naive algorithm using repeated addition. The result is then 
stored into memory as well.

\bigskip
\bigskip

\subsection*{3. Group reflection}

Anthony: I believe our collaboration as a team for Part II has improved tremendously since Part I of the group project. Using VSCode 
LiveShare to work on the project together was definitely effective, and I am very happy with my group as everyone is extremely 
hardworking and motivated for the group project and we would work on the project for many hours every day. An improvement for 
next time would be to talk about the structure of the structure of the project as a whole, especially how the later parts will be implemented as 
a problem we realized later is that we had to change the implementation of some of our code from earlier parts, so I think better 
planning in advance instead of delving right into the implementation would be a good idea for next time. 
\\ \\
Ezra: Team communication has definitely improved from the assembler section. As before, the use of VSCode LiveShare allowed multiple
people to work on the same problem at the same time - effectively increasing the productivity of our group. This was done with the use
of voice calls and text chat in the live coding sessions. Furthermore, we decided to do better planning before starting out our project. 
The plan described the map data structure we would use as well as the general functioning of the assembler. We discussed various parts, such
as whether to do a single-pass or two-pass method and how the symbol table would be implemented and used, in order to minimise unnecessary time
discussing these things later as it would break up the flow of productivity. To add to this, we frequently tested small chunks of the program
instead of leaving it all up to the very end. As a result, I would push to use a live sharing option for future group projects as 
well as a reasonable amount of time dedicated to planning the project as opposed to jumping straight in. However, our planning
did not anticipate minute inconveniences in the project. Since we did the project sequentially, we often discovered problems that affects previous
parts of the program. I think this problem could have been resolved if we split up and did sections in a non-linear fashion
so that if such an issue appeared, this could be discussed by the group and be resolved quicker as well as making a plan of the project as a whole.
\\ \\
Michalis: As I've said in the previous report, I believe that our team worked incredibly well. The fact that 
we were using VSCode LiveShare mode allowed us to help each other and learn a lot of things at the same time. 
Moreover, I believe that in Part II we further improved our communication compared to Part I. As a result, progress was
done much quicker since we didn't cluster up and work on the same part of the project. Instead, we split the tasks between us so we could
work on several parts but made sure that someone was available if another person needed help. Finally, this group project was beneficial 
for me since it was my first major group project that took several weeks of time and taught me a lot of things about how a team should 
operate efficiently. For the next time I work in a group I will definitely follow some of the tactics that we implemented on this project.
Firstly, I will make sure that we split the tasks so we work more efficiently, but also give the required importance on communication and
feedback to each other. What I've seen from this project is that communicating with each other constantly and informing every team member
for each major change we've made is crucial in order to keep them updated.
\\ \\
Sarveen: I think that our teamwork has improved further compared to the emulator phase of the project. I believe that
by using our approach of live-coding at the same time as opposed to coding seperately and merging has significantly
decreased the amount of time it took to complete the project. This is due to the lack of any serious conflicts we could
have had were we to use git, and also because we were able to communicate in real time. For instance, this allowed us 
to clarify parts of the specification and continue coding without delay as the live chat and voice chat is embedded
into the platform. However, there is still some room for improvement. The most important thing that I think we could 
improve on is to set miniature goals that we should aim to complete in a given time. I think this will be beneficial 
as it will allow us to have more structure in the way we complete the project and we may be able to work faster if we
have small achievable deadlines. Another important strategy that we should consider is to get some team members to work
on later parts of the project instead of doing it sequentially from the beginning. The reason for this is that we had 
to rewrite sections of our code to accommodate simpler functions in later parts of our code. For example, we wrote the 
map first but then we had to modify it once we got to a section of single data transfer. 

\bigskip
\bigskip

\subsection*{4. Individual reflection}

Anthony: This is the first time we have worked on a large programming group project, so I was scared about how we would be 
able to split the group work evenly and effectively. However once we got on with the project we slowly gained experience 
each day and we would be more used to the aspect of working together as a group. A weakness I found for me was that it 
takes me a while to get a grasp an idea and get started on how to implement an idea from the spec but after I understood 
the concept of the implementation, I am able to work through it. Another strength I found is my ability to spot bugs or 
mistakes in code, which is a very tedious and time-consuming task. I am very happy that I got the experience to work in a 
programming group project like this as it taught me how to work together in a team, and how to share and listen to each other.
\\ \\
Ezra: I think the hardest part of group work is starting out with new people. Since the 4 of us have never undertaken a 
large programming group project, and with the additional fact that none of us had programmed in C, it seemed like it would
be a very difficult task. Nevertheless, I think that our group worked very well together. I think one of my strengths was 
communication and getting the group to start working in the beginning. Another strength I think I have is thinking algorithmically - 
I think I was quite helpful when other group members were asking how to implement or fix something. I think one of my weaknesses is 
taking too long in trying to do a single task that is difficult - looking back, I think it would be better for me to work on the parts 
that may come after it and then work on the first problem with somebody else.
\\ \\
Michalis: Initially, when I learned about the group project I got pretty worried because I am a pretty shy person, which 
I thought would have a significant impact on my communication with the team. Moreover, I thought that I wouldn't have the courage 
to express my own opinions and ideas. On the other side, I consider myself a very organized person which I believe is 
a significant strength in a group project, since I like putting tasks in order, focusing on one thing at a time and always 
having a plan about what I am currently doing and what I want to do and achieve every day. However, after we started the 
project my worries thankfully didn't come true. Thanks to my team and our chemistry I felt really relaxed
from the start and I was always saying my opinion without thinking negatively of it. On the other hand, one weakness
that I discovered through this project is that sometimes I am too confident about my ideas and I really want them to get implemented. 
However, I think that through this project I learned to be more open to other ideas, understand when they are better than mine or how they
can complement my ideas and create the most optimal one. Finally, one strength that I've discovered is my persistence to make all parts
look good, uniform and consistent, which is necessary in all group projects.
\\ \\
Sarveen: Prior to this project, I thought that it may be difficult to work in a group because of my lack of experience in teamwork.
I thought that my main weakness would be my communiation skills. However, it soon became apparent that there was no issue and I was able
to both communicate my ideas well and also co-ordinate with others. One weakness that I did find to have was understanding certain
details of the specification. For example, when it came to setting the CPSR flags for the emulator, I had to ponder over it for a long
time before I was able to implement it. Although this was the case, I was able to spend less and less time as we worked through the 
project and I was able to understand the specification for the assembler fairly quickly. One strength I think I have is my resilience as 
I am able to keep working on a difficult problem until I solve it. 
\end{document}